
\documentclass[9pt]{article}

\usepackage[a4paper,margin=2cm,top=3cm]{geometry}

% graphics
\usepackage{tikz}
\usepackage{graphicx}

% math
\usepackage{amsmath}
\usepackage{amsfonts}

\usepackage[hidelinks]{hyperref}

% header
\usepackage{fancyhdr}
 
\fancyhf{}
\lhead{\includegraphics[width=1cm]{sharif-logo}}
\rhead{تمرینِ نخست}
\chead{مبانیِ برنامه‌نویسی}
\cfoot{\thepage}
 

\usepackage[extrafootnotefeatures,localise]{xepersian}
\settextfont{IRNazanin}


\گزینش‌قلم‌لاتین‌متن[Scale=0.8]{Liberation Serif}
\گزینش‌قلم‌اعدادفرمولها{IRNazanin}

\شروع{نوشتار}
\begin {center}
{\small \hspace*{-2mm} به نام خدا}
\end{center}
\vspace*{80mm}

\thispagestyle{empty}
\شروع{وسط‌چین}

\درج‌تصویر[width=3cm]{sharif-logo}\\
\شروع{درشت‌درشت}
{\huge
درس مبانی برنامه‌سازی \\
}\bigskip{\large
نیم‌سال اول 96-97\\
\vspace*{2mm}
دانشکدهٔ مهندسی کامپیوتر \\
\hspace*{-3mm}
دانشگاه صنعتی شریف \\
}
\پایان{درشت‌درشت}
\پایان{وسط‌چین}
\پرش‌بلند

\bigskip

\hrule
\begin{center}
\vspace{6mm}
{\large \hspace*{6mm}
مدرس 
{\bf  \hspace*{3mm}
مهران ریواده \\
}\vspace{1mm}  
تمرین
{\bf  \hspace*{3mm}
نخست\\
}\vspace{1mm} \hspace*{-6mm}
مبحث
{\bf  \hspace*{3mm}
IO\\
}\vspace{1mm}\hspace*{2mm}
موعد تحویل
{\bf  \hspace*{3mm}
15 آبان 1396\\
}
}


\پرش‌بلند


\شروع{درشت}
\شروع{فقرات}
\فقره{پاسخ این‌تمرین را در قالب یک‌فایلِ\متن‌لاتین{pdf} در کوئرا آپلود کنید}
\فقره{در صورت مشاهده‌ی هرگونه تقلب، نمره‌ی هر دو نفر -100 در نظر گرفته خواهد شد.}
\فقره{به ازای هر روز دیرکرد در بارگذاری تمرین‌ها، 10 درصد جریمه منظور خواهد شد.}
\فقره{در صورت بروز ابهام در مورد سوالات، می‌توانید حداکثر تا 24 ساعت قبل از موعد تحویل، سوالات خود را در سایت کوئرا بپرسید.}
\پایان{فقرات}
\پایان{درشت}
\end{center}
\صفحه‌شکن
\سبک‌صفحه{fancy}

سؤال نخست

\پایان{نوشتار}
